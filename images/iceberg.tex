\begingroup
\sethlcolor{pink}%
\setlength{\fboxsep}{10pt}%
% Setting up the verbatim boxes with code examples.
\def\linenumber{%
  \parbox{8pt}{%
    \footnotesize
    \raggedleft
    \textnormal{\arabic{VerbboxLineNo}}%
  }%
  \hspace{5pt}%
}%
\def\startonline#1{%
  \ifnum\value{VerbboxLineNo}=1\relax
    \addtocounter{VerbboxLineNo}{#1}%
    \addtocounter{VerbboxLineNo}{-1}%
  \fi
}%
\verbfilebox[\linenumber\small]{images/iceberg-code-01.tex}%
\newsavebox\firstcode\sbox\firstcode{\box\savedverbbox}%
\addtocounter{VerbboxLineNo}{2}%
\verbfilebox[\startonline{3}\linenumber\small]{images/iceberg-code-02.tex}%
\newsavebox\secondcode\sbox\secondcode{\box\savedverbbox}%
\verbfilebox[\startonline{11}\linenumber\small]{images/iceberg-code-03.tex}%
\newsavebox\thirdcode\sbox\thirdcode{\box\savedverbbox}%
\verbfilebox[\color{red}\def\color##1{}\startonline{15}\linenumber\small]{images/iceberg-code-04.tex}%
\newsavebox\fourthcode\sbox\fourthcode{\box\savedverbbox}%
\begin{tikzpicture}[every node/.style={inner sep=0pt, outer sep=0pt}]
% The iceberg in the background
\node (background) at (0, 0) {{%
  \transparent{0.15}%
  \includegraphics[width=0.7\linewidth]{images/iceberg-color}%
}};
% The water surface
\draw[decorate, decoration={snake, segment length=30.10mm, amplitude=4mm}, yshift=-5.5pt] (-10.1, 5) -- (-8, 5);
\draw[decorate, decoration={snake, segment length=15.05mm, amplitude=2mm}] (-8, 5) -- (4, 5);
\draw[decorate, decoration={snake, segment length=30.10mm, amplitude=4mm}, yshift=-5.5pt] (4.4, 5) -- (6.3, 5);
% The LaTeX source code as an abstract ice floe
\node [left=of background.north, xshift=0.8cm, yshift=-1.5cm] (code-01-tex) {%
  \fcolorbox{black}{white}{\usebox{\firstcode}}%
};
% The result of compiling the source code as another ice floe
\node [right=12pt of code-01-tex.south east, anchor=south west, xshift=1cm, rotate=-1] (code-01-pdf) {%
  \fcolorbox{black}{white}{%
    \includegraphics[width=3cm]{images/iceberg-code-01}%
  }%
};
% The compilation, i.e. the "above-the-surface" of a TeX document.
\draw [-{Stealth[length=3mm, width=3mm]}, line width=\fboxrule] ([xshift=5pt] code-01-tex.east) to [bend left] node [text width=5cm, midway, above, align=center, xshift=35pt, yshift=20pt] {%
  compile with\\
  \texttt{lualatex example.tex}%
} ([yshift=5pt] code-01-pdf.north);
% The result of the preprocessing a TeX document with explcheck.
%% Preprocessing
\node [below=60pt of code-01-tex.south west, anchor=north west, xshift=-13pt] (code-02-tex) {%
  \fcolorbox{black}{white}{\usebox{\secondcode}}%
};
\node [below=8pt of code-02-tex.south west, anchor=north west] (code-03-tex) {%
  \fcolorbox{black}{white}{\usebox{\thirdcode}}%
};
\node [below=8pt of code-02-tex.south east, anchor=north east, xshift=-\dimexpr(2\fboxsep-6pt+2\fboxrule)] (code-04-tex) {%
  \begingroup
  \fboxsep=3pt%
  \fboxrule=0pt%
  \fcolorbox{red}{pink}{\usebox{\fourthcode}}%
  \endgroup
};
\node [below=70pt of code-02-tex.south, anchor=north, text width=9cm, align=center] {%
  \small
  (a) Preprocessing identifies two expl3 parts and warns about an unnecessary command \cs{ExplSyntaxOff} on line 15.
};
\draw [-{Stealth[length=3mm, width=3mm]}, line width=\fboxrule] ([xshift=-2cm, yshift=-5pt] code-01-tex.south) to [bend right] node [text width=5cm, midway, right, align=left, xshift=12.5pt, yshift=-10pt] {%
  analyze with\\
  \texttt{explcheck example.tex}\\
} ([xshift=-2cm, yshift=5pt] code-02-tex.north);
% Lexical analysis
\def\cshelper#1#2#3{\fcolorbox{#2}{#3}{\vphantom{/}\texttt{#1}} }%
\def\cs#1{\cshelper{#1}{black}{white}}%
\def\errorcs#1{\textcolor{red}{\cshelper{#1}{red}{pink}}}%
\def\token#1#2{\tokenhelper{#1}{#2} }%
\def\errortoken#1#2{%
  \begingroup
  \fboxsep=0pt%
  \fboxrule=0pt%
  \fcolorbox{red}{pink}{\textcolor{red}{\tokenhelper{\strut#1}{#2}}} 
  \endgroup
}%
\def\tokenhelper#1#2{\texttt{#1}\textsubscript{#2}}%
\node [right=25pt of code-02-tex.north east, anchor=north west, text width=5.95cm, align=left, yshift=38pt] (code-05-tex) {%
  \begingroup
    \setlength{\fboxsep}{2pt}%
    \small
    \baselineskip=13pt%
    \cs{iow\_new:N}%
    \cs{l\_example\_iow}%
    \cs{iow\_open:Nn}%
    \cs{l\_example\_iow}%
    \token{\{}{1}%
    \token{e}{11}%
    \token{x}{11}%
    \token{a}{11}%
    \token{m}{11}%
    \token{p}{11}%
    \token{l}{11}%
    \token{e}{11}%
    \token{.}{12}%
    \token{i}{11}%
    \token{d}{11}%
    \token{x}{11}%
    \token{\}}{2}%
    \cs{cs\_new:Nn}%
    \cs{\_\_example\_foo:n}%
    \token{\{}{1}%
    \cs{cs\_new:Nn}%
    \cs{example\_bar:nn}%
    \errortoken{\{}{1}%
    \errortoken{\#}{6}%
    \token{1}{12}%
    \token{,}{12}%
    \token{\textvisiblespace}{10}%
    \token{\#}{6}%
    \token{\#}{6}%
    \token{1}{12}%
    \token{\#}{6}%
    \token{\#}{6}%
    \token{2}{12}%
    \errortoken{!}{12}%
    \errortoken{\}}{2}%
    \token{\}}{2}%
    \cs{cs\_generate\_variant:Nn}%
    \cs{\_\_example\_foo:n}%
    \token{\{}{1}%
    \token{V}{11}%
    \token{,}{12}%
    \token{x}{11}%
    \token{\}}{2}%
    \par
  \endgroup
};
\node [below=8pt of code-05-tex.south west, anchor=north west, text width=5.95cm, align=left] (code-06-tex) {%
  \begingroup
    \setlength{\fboxsep}{2pt}%
    \small
    \baselineskip=13pt%
    \cs{\_\_example\_foo:x}%
    \token{\{}{1}%
    \token{H}{11}%
    \token{e}{11}%
    \token{l}{11}%
    \token{l}{11}%
    \token{o}{11}%
    \token{\}}{2}%
    \errorcs{begin}%
    \errortoken{\{}{1}%
    \errortoken{d}{11}%
    \token{o}{11}%
    \token{c}{11}%
    \token{u}{11}%
    \token{m}{11}%
    \token{e}{11}%
    \token{n}{11}%
    \errortoken{t}{11}%
    \errortoken{\}}{2}%
    \cs{example\_bar:nn}%
    \token{w}{11}%
    \token{\{}{1}%
    \token{o}{11}%
    \token{r}{11}%
    \token{l}{11}%
    \token{d}{11}%
    \token{\}}{2}%
    \par
  \endgroup
};
\node [below=10pt of code-06-tex.south, anchor=north, text width=6cm, align=center, xshift=-3pt] (code-07-tex) {%
  \small
  (b) Lexical analysis identifies 68 tokens and warns about missing whitespaces around braces (\texttt{\{}, \texttt{\}}) on lines 6 and 12.
};
%\draw [-{Stealth[length=2mm, width=2mm]}, line width=\fboxrule] ([xshift=-5pt, yshift=5pt] code-02-tex.north east) to [bend left] ([xshift=-5pt, yshift=-5pt] code-05-tex.north west);
%\draw [-{Stealth[length=2mm, width=2mm]}, line width=\fboxrule] ([xshift=5pt, yshift=5pt] code-03-tex.south east) to [bend right] ([xshift=-5pt, yshift=-8pt] code-06-tex.north west);
% Syntactic analysis
\node [below=40pt of code-03-tex.south west, anchor=north west, text width=7.6cm, align=left, xshift=-5pt] (code-07-tex) {%
  \small
  \begin{enumerate}
  \item[C1.] Call \texttt{\textbackslash iow\_new:N} with the argument \texttt{\textbackslash l\_example\_iow}.
  \item[C2.] Call \texttt{\textbackslash iow\_open:Nn} with arguments \texttt{\textbackslash l\_example\_iow} and \{ \texttt{example.idx} \}.
  \item[C3.] Call \texttt{\textbackslash cs\_new:Nn} with arguments \texttt{\textbackslash \_\_example\_foo:n} and \texttt{\{ \textbackslash cs\_new:Nn \ldots\ \}}.
  \item[C4.] Call \texttt{\textbackslash cs\_generate\_variant:Nn} with \texttt{\textbackslash \_\_example\_foo:n} and \texttt{\{ V, x \}}.
  \item[C5.] Call \texttt{\textbackslash \_\_example\_foo:x} with \texttt{\{ Hello \}}.
  \item[C6.] Call \texttt{\textbackslash example\_bar:nn} with \textcolor{red}{\hl{\texttt{w}}} and \texttt{\{ orld \}}.
  \end{enumerate}
};
\node [below=10pt of code-07-tex.south, anchor=north, text width=7.2cm, align=center, xshift=-5pt] (code-07-label) {%
  \small
  (c) Syntactic analysis identifies six expl3 calls and warns about missing braces around the \mbox{\texttt{n}-type} argument~\cite[Section~1.1]{latex2025interface3} \texttt{w} on line 13.
};
% Semantic analysis
\node [right=12.5pt of code-07-tex.south east, anchor=south west, text width=6.1cm, align=left] (code-08-tex) {%
  \small
  \begin{enumerate}[nosep, itemsep=1pt]
  \item[S1.] \textcolor{red}{\hl{Reserve a stream \texttt{\textbackslash l\_example\_iow}.}}
  \item[S2.] Open a file \texttt{example.idx} for writing.
  \item[S3.] Define a function \texttt{\textbackslash \_\_example\_foo:n}:
  \begin{enumerate}
  \item[S4.] Define \texttt{\textbackslash \_\_example\_bar:nn} with text \meta{argument}\texttt{,\textvisiblespace}\meta{arguments}\texttt{!}.
  \end{enumerate}
  \item[S5.] \textcolor{red}{\hl{Generate variant \texttt{\textbackslash \_\_example\_foo:V}.}}
  \item[S6.] Generate variant \texttt{\textbackslash \_\_example\_foo:x}.
  \item[S7.] Call \texttt{\textbackslash \_\_example\_foo:x}.
  \item[S8.] Call \texttt{\textbackslash \_\_example\_bar:nn}.
  \end{enumerate}
};
% Flow analysis
\draw[decorate, decoration={brace, amplitude=47.5pt}] ([yshift=-5pt] code-08-tex.north east) -- ([yshift=25pt] code-08-tex.south east) node[midway, xshift=57.5pt] {\small Ch1};
\draw[decorate, decoration={brace, amplitude=5pt}] ([yshift=-37.5pt] code-08-tex.north east) -- ([yshift=47.5pt] code-08-tex.south east) node[midway, xshift=15pt] {\small Ch2};
\draw [-{Stealth[length=2mm, width=2mm, open]}, line width=\fboxrule] ([xshift=-53pt, yshift=17pt] code-08-tex.south east) to ([xshift=25pt, yshift=17pt] code-08-tex.south east);
\node at ([xshift=35pt, yshift=17pt] code-08-tex.south east) {\small Ch3};
\draw [-{Stealth[length=2mm, width=2mm, open]}, line width=\fboxrule] ([xshift=-48pt, yshift=5pt] code-08-tex.south east) to ([xshift=25pt, yshift=5pt] code-08-tex.south east);
\node at ([xshift=35pt, yshift=5pt] code-08-tex.south east) {\small Ch4};
% Label for both semantic and flow analyses.
\node [right=6pt of code-07-label.north east, anchor=north west, text width=9cm, align=center] (code-08-label) {%
  \small
  (d) Semantic analysis identifies eight statements and warns about an unused variant from line 7. Flow analysis identifies four chunks and warns about an unclosed stream from line 4.
};
\end{tikzpicture}
\endgroup
\vspace{1pt}%